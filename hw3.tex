
% Default to the notebook output style

    


% Inherit from the specified cell style.




    
\documentclass[11pt]{article}

    
    
    \usepackage[T1]{fontenc}
    % Nicer default font (+ math font) than Computer Modern for most use cases
    \usepackage{mathpazo}

    % Basic figure setup, for now with no caption control since it's done
    % automatically by Pandoc (which extracts ![](path) syntax from Markdown).
    \usepackage{graphicx}
    % We will generate all images so they have a width \maxwidth. This means
    % that they will get their normal width if they fit onto the page, but
    % are scaled down if they would overflow the margins.
    \makeatletter
    \def\maxwidth{\ifdim\Gin@nat@width>\linewidth\linewidth
    \else\Gin@nat@width\fi}
    \makeatother
    \let\Oldincludegraphics\includegraphics
    % Set max figure width to be 80% of text width, for now hardcoded.
    \renewcommand{\includegraphics}[1]{\Oldincludegraphics[width=.8\maxwidth]{#1}}
    % Ensure that by default, figures have no caption (until we provide a
    % proper Figure object with a Caption API and a way to capture that
    % in the conversion process - todo).
    \usepackage{caption}
    \DeclareCaptionLabelFormat{nolabel}{}
    \captionsetup{labelformat=nolabel}

    \usepackage{adjustbox} % Used to constrain images to a maximum size 
    \usepackage{xcolor} % Allow colors to be defined
    \usepackage{enumerate} % Needed for markdown enumerations to work
    \usepackage{geometry} % Used to adjust the document margins
    \usepackage{amsmath} % Equations
    \usepackage{amssymb} % Equations
    \usepackage{textcomp} % defines textquotesingle
    % Hack from http://tex.stackexchange.com/a/47451/13684:
    \AtBeginDocument{%
        \def\PYZsq{\textquotesingle}% Upright quotes in Pygmentized code
    }
    \usepackage{upquote} % Upright quotes for verbatim code
    \usepackage{eurosym} % defines \euro
    \usepackage[mathletters]{ucs} % Extended unicode (utf-8) support
    \usepackage[utf8x]{inputenc} % Allow utf-8 characters in the tex document
    \usepackage{fancyvrb} % verbatim replacement that allows latex
    \usepackage{grffile} % extends the file name processing of package graphics 
                         % to support a larger range 
    % The hyperref package gives us a pdf with properly built
    % internal navigation ('pdf bookmarks' for the table of contents,
    % internal cross-reference links, web links for URLs, etc.)
    \usepackage{hyperref}
    \usepackage{longtable} % longtable support required by pandoc >1.10
    \usepackage{booktabs}  % table support for pandoc > 1.12.2
    \usepackage[inline]{enumitem} % IRkernel/repr support (it uses the enumerate* environment)
    \usepackage[normalem]{ulem} % ulem is needed to support strikethroughs (\sout)
                                % normalem makes italics be italics, not underlines
    

    
    
    % Colors for the hyperref package
    \definecolor{urlcolor}{rgb}{0,.145,.698}
    \definecolor{linkcolor}{rgb}{.71,0.21,0.01}
    \definecolor{citecolor}{rgb}{.12,.54,.11}

    % ANSI colors
    \definecolor{ansi-black}{HTML}{3E424D}
    \definecolor{ansi-black-intense}{HTML}{282C36}
    \definecolor{ansi-red}{HTML}{E75C58}
    \definecolor{ansi-red-intense}{HTML}{B22B31}
    \definecolor{ansi-green}{HTML}{00A250}
    \definecolor{ansi-green-intense}{HTML}{007427}
    \definecolor{ansi-yellow}{HTML}{DDB62B}
    \definecolor{ansi-yellow-intense}{HTML}{B27D12}
    \definecolor{ansi-blue}{HTML}{208FFB}
    \definecolor{ansi-blue-intense}{HTML}{0065CA}
    \definecolor{ansi-magenta}{HTML}{D160C4}
    \definecolor{ansi-magenta-intense}{HTML}{A03196}
    \definecolor{ansi-cyan}{HTML}{60C6C8}
    \definecolor{ansi-cyan-intense}{HTML}{258F8F}
    \definecolor{ansi-white}{HTML}{C5C1B4}
    \definecolor{ansi-white-intense}{HTML}{A1A6B2}

    % commands and environments needed by pandoc snippets
    % extracted from the output of `pandoc -s`
    \providecommand{\tightlist}{%
      \setlength{\itemsep}{0pt}\setlength{\parskip}{0pt}}
    \DefineVerbatimEnvironment{Highlighting}{Verbatim}{commandchars=\\\{\}}
    % Add ',fontsize=\small' for more characters per line
    \newenvironment{Shaded}{}{}
    \newcommand{\KeywordTok}[1]{\textcolor[rgb]{0.00,0.44,0.13}{\textbf{{#1}}}}
    \newcommand{\DataTypeTok}[1]{\textcolor[rgb]{0.56,0.13,0.00}{{#1}}}
    \newcommand{\DecValTok}[1]{\textcolor[rgb]{0.25,0.63,0.44}{{#1}}}
    \newcommand{\BaseNTok}[1]{\textcolor[rgb]{0.25,0.63,0.44}{{#1}}}
    \newcommand{\FloatTok}[1]{\textcolor[rgb]{0.25,0.63,0.44}{{#1}}}
    \newcommand{\CharTok}[1]{\textcolor[rgb]{0.25,0.44,0.63}{{#1}}}
    \newcommand{\StringTok}[1]{\textcolor[rgb]{0.25,0.44,0.63}{{#1}}}
    \newcommand{\CommentTok}[1]{\textcolor[rgb]{0.38,0.63,0.69}{\textit{{#1}}}}
    \newcommand{\OtherTok}[1]{\textcolor[rgb]{0.00,0.44,0.13}{{#1}}}
    \newcommand{\AlertTok}[1]{\textcolor[rgb]{1.00,0.00,0.00}{\textbf{{#1}}}}
    \newcommand{\FunctionTok}[1]{\textcolor[rgb]{0.02,0.16,0.49}{{#1}}}
    \newcommand{\RegionMarkerTok}[1]{{#1}}
    \newcommand{\ErrorTok}[1]{\textcolor[rgb]{1.00,0.00,0.00}{\textbf{{#1}}}}
    \newcommand{\NormalTok}[1]{{#1}}
    
    % Additional commands for more recent versions of Pandoc
    \newcommand{\ConstantTok}[1]{\textcolor[rgb]{0.53,0.00,0.00}{{#1}}}
    \newcommand{\SpecialCharTok}[1]{\textcolor[rgb]{0.25,0.44,0.63}{{#1}}}
    \newcommand{\VerbatimStringTok}[1]{\textcolor[rgb]{0.25,0.44,0.63}{{#1}}}
    \newcommand{\SpecialStringTok}[1]{\textcolor[rgb]{0.73,0.40,0.53}{{#1}}}
    \newcommand{\ImportTok}[1]{{#1}}
    \newcommand{\DocumentationTok}[1]{\textcolor[rgb]{0.73,0.13,0.13}{\textit{{#1}}}}
    \newcommand{\AnnotationTok}[1]{\textcolor[rgb]{0.38,0.63,0.69}{\textbf{\textit{{#1}}}}}
    \newcommand{\CommentVarTok}[1]{\textcolor[rgb]{0.38,0.63,0.69}{\textbf{\textit{{#1}}}}}
    \newcommand{\VariableTok}[1]{\textcolor[rgb]{0.10,0.09,0.49}{{#1}}}
    \newcommand{\ControlFlowTok}[1]{\textcolor[rgb]{0.00,0.44,0.13}{\textbf{{#1}}}}
    \newcommand{\OperatorTok}[1]{\textcolor[rgb]{0.40,0.40,0.40}{{#1}}}
    \newcommand{\BuiltInTok}[1]{{#1}}
    \newcommand{\ExtensionTok}[1]{{#1}}
    \newcommand{\PreprocessorTok}[1]{\textcolor[rgb]{0.74,0.48,0.00}{{#1}}}
    \newcommand{\AttributeTok}[1]{\textcolor[rgb]{0.49,0.56,0.16}{{#1}}}
    \newcommand{\InformationTok}[1]{\textcolor[rgb]{0.38,0.63,0.69}{\textbf{\textit{{#1}}}}}
    \newcommand{\WarningTok}[1]{\textcolor[rgb]{0.38,0.63,0.69}{\textbf{\textit{{#1}}}}}
    
    
    % Define a nice break command that doesn't care if a line doesn't already
    % exist.
    \def\br{\hspace*{\fill} \\* }
    % Math Jax compatability definitions
    \def\gt{>}
    \def\lt{<}
    % Document parameters
    \title{hw3}
    
    
    

    % Pygments definitions
    
\makeatletter
\def\PY@reset{\let\PY@it=\relax \let\PY@bf=\relax%
    \let\PY@ul=\relax \let\PY@tc=\relax%
    \let\PY@bc=\relax \let\PY@ff=\relax}
\def\PY@tok#1{\csname PY@tok@#1\endcsname}
\def\PY@toks#1+{\ifx\relax#1\empty\else%
    \PY@tok{#1}\expandafter\PY@toks\fi}
\def\PY@do#1{\PY@bc{\PY@tc{\PY@ul{%
    \PY@it{\PY@bf{\PY@ff{#1}}}}}}}
\def\PY#1#2{\PY@reset\PY@toks#1+\relax+\PY@do{#2}}

\expandafter\def\csname PY@tok@gd\endcsname{\def\PY@tc##1{\textcolor[rgb]{0.63,0.00,0.00}{##1}}}
\expandafter\def\csname PY@tok@gu\endcsname{\let\PY@bf=\textbf\def\PY@tc##1{\textcolor[rgb]{0.50,0.00,0.50}{##1}}}
\expandafter\def\csname PY@tok@gt\endcsname{\def\PY@tc##1{\textcolor[rgb]{0.00,0.27,0.87}{##1}}}
\expandafter\def\csname PY@tok@gs\endcsname{\let\PY@bf=\textbf}
\expandafter\def\csname PY@tok@gr\endcsname{\def\PY@tc##1{\textcolor[rgb]{1.00,0.00,0.00}{##1}}}
\expandafter\def\csname PY@tok@cm\endcsname{\let\PY@it=\textit\def\PY@tc##1{\textcolor[rgb]{0.25,0.50,0.50}{##1}}}
\expandafter\def\csname PY@tok@vg\endcsname{\def\PY@tc##1{\textcolor[rgb]{0.10,0.09,0.49}{##1}}}
\expandafter\def\csname PY@tok@vi\endcsname{\def\PY@tc##1{\textcolor[rgb]{0.10,0.09,0.49}{##1}}}
\expandafter\def\csname PY@tok@vm\endcsname{\def\PY@tc##1{\textcolor[rgb]{0.10,0.09,0.49}{##1}}}
\expandafter\def\csname PY@tok@mh\endcsname{\def\PY@tc##1{\textcolor[rgb]{0.40,0.40,0.40}{##1}}}
\expandafter\def\csname PY@tok@cs\endcsname{\let\PY@it=\textit\def\PY@tc##1{\textcolor[rgb]{0.25,0.50,0.50}{##1}}}
\expandafter\def\csname PY@tok@ge\endcsname{\let\PY@it=\textit}
\expandafter\def\csname PY@tok@vc\endcsname{\def\PY@tc##1{\textcolor[rgb]{0.10,0.09,0.49}{##1}}}
\expandafter\def\csname PY@tok@il\endcsname{\def\PY@tc##1{\textcolor[rgb]{0.40,0.40,0.40}{##1}}}
\expandafter\def\csname PY@tok@go\endcsname{\def\PY@tc##1{\textcolor[rgb]{0.53,0.53,0.53}{##1}}}
\expandafter\def\csname PY@tok@cp\endcsname{\def\PY@tc##1{\textcolor[rgb]{0.74,0.48,0.00}{##1}}}
\expandafter\def\csname PY@tok@gi\endcsname{\def\PY@tc##1{\textcolor[rgb]{0.00,0.63,0.00}{##1}}}
\expandafter\def\csname PY@tok@gh\endcsname{\let\PY@bf=\textbf\def\PY@tc##1{\textcolor[rgb]{0.00,0.00,0.50}{##1}}}
\expandafter\def\csname PY@tok@ni\endcsname{\let\PY@bf=\textbf\def\PY@tc##1{\textcolor[rgb]{0.60,0.60,0.60}{##1}}}
\expandafter\def\csname PY@tok@nl\endcsname{\def\PY@tc##1{\textcolor[rgb]{0.63,0.63,0.00}{##1}}}
\expandafter\def\csname PY@tok@nn\endcsname{\let\PY@bf=\textbf\def\PY@tc##1{\textcolor[rgb]{0.00,0.00,1.00}{##1}}}
\expandafter\def\csname PY@tok@no\endcsname{\def\PY@tc##1{\textcolor[rgb]{0.53,0.00,0.00}{##1}}}
\expandafter\def\csname PY@tok@na\endcsname{\def\PY@tc##1{\textcolor[rgb]{0.49,0.56,0.16}{##1}}}
\expandafter\def\csname PY@tok@nb\endcsname{\def\PY@tc##1{\textcolor[rgb]{0.00,0.50,0.00}{##1}}}
\expandafter\def\csname PY@tok@nc\endcsname{\let\PY@bf=\textbf\def\PY@tc##1{\textcolor[rgb]{0.00,0.00,1.00}{##1}}}
\expandafter\def\csname PY@tok@nd\endcsname{\def\PY@tc##1{\textcolor[rgb]{0.67,0.13,1.00}{##1}}}
\expandafter\def\csname PY@tok@ne\endcsname{\let\PY@bf=\textbf\def\PY@tc##1{\textcolor[rgb]{0.82,0.25,0.23}{##1}}}
\expandafter\def\csname PY@tok@nf\endcsname{\def\PY@tc##1{\textcolor[rgb]{0.00,0.00,1.00}{##1}}}
\expandafter\def\csname PY@tok@si\endcsname{\let\PY@bf=\textbf\def\PY@tc##1{\textcolor[rgb]{0.73,0.40,0.53}{##1}}}
\expandafter\def\csname PY@tok@s2\endcsname{\def\PY@tc##1{\textcolor[rgb]{0.73,0.13,0.13}{##1}}}
\expandafter\def\csname PY@tok@nt\endcsname{\let\PY@bf=\textbf\def\PY@tc##1{\textcolor[rgb]{0.00,0.50,0.00}{##1}}}
\expandafter\def\csname PY@tok@nv\endcsname{\def\PY@tc##1{\textcolor[rgb]{0.10,0.09,0.49}{##1}}}
\expandafter\def\csname PY@tok@s1\endcsname{\def\PY@tc##1{\textcolor[rgb]{0.73,0.13,0.13}{##1}}}
\expandafter\def\csname PY@tok@dl\endcsname{\def\PY@tc##1{\textcolor[rgb]{0.73,0.13,0.13}{##1}}}
\expandafter\def\csname PY@tok@ch\endcsname{\let\PY@it=\textit\def\PY@tc##1{\textcolor[rgb]{0.25,0.50,0.50}{##1}}}
\expandafter\def\csname PY@tok@m\endcsname{\def\PY@tc##1{\textcolor[rgb]{0.40,0.40,0.40}{##1}}}
\expandafter\def\csname PY@tok@gp\endcsname{\let\PY@bf=\textbf\def\PY@tc##1{\textcolor[rgb]{0.00,0.00,0.50}{##1}}}
\expandafter\def\csname PY@tok@sh\endcsname{\def\PY@tc##1{\textcolor[rgb]{0.73,0.13,0.13}{##1}}}
\expandafter\def\csname PY@tok@ow\endcsname{\let\PY@bf=\textbf\def\PY@tc##1{\textcolor[rgb]{0.67,0.13,1.00}{##1}}}
\expandafter\def\csname PY@tok@sx\endcsname{\def\PY@tc##1{\textcolor[rgb]{0.00,0.50,0.00}{##1}}}
\expandafter\def\csname PY@tok@bp\endcsname{\def\PY@tc##1{\textcolor[rgb]{0.00,0.50,0.00}{##1}}}
\expandafter\def\csname PY@tok@c1\endcsname{\let\PY@it=\textit\def\PY@tc##1{\textcolor[rgb]{0.25,0.50,0.50}{##1}}}
\expandafter\def\csname PY@tok@fm\endcsname{\def\PY@tc##1{\textcolor[rgb]{0.00,0.00,1.00}{##1}}}
\expandafter\def\csname PY@tok@o\endcsname{\def\PY@tc##1{\textcolor[rgb]{0.40,0.40,0.40}{##1}}}
\expandafter\def\csname PY@tok@kc\endcsname{\let\PY@bf=\textbf\def\PY@tc##1{\textcolor[rgb]{0.00,0.50,0.00}{##1}}}
\expandafter\def\csname PY@tok@c\endcsname{\let\PY@it=\textit\def\PY@tc##1{\textcolor[rgb]{0.25,0.50,0.50}{##1}}}
\expandafter\def\csname PY@tok@mf\endcsname{\def\PY@tc##1{\textcolor[rgb]{0.40,0.40,0.40}{##1}}}
\expandafter\def\csname PY@tok@err\endcsname{\def\PY@bc##1{\setlength{\fboxsep}{0pt}\fcolorbox[rgb]{1.00,0.00,0.00}{1,1,1}{\strut ##1}}}
\expandafter\def\csname PY@tok@mb\endcsname{\def\PY@tc##1{\textcolor[rgb]{0.40,0.40,0.40}{##1}}}
\expandafter\def\csname PY@tok@ss\endcsname{\def\PY@tc##1{\textcolor[rgb]{0.10,0.09,0.49}{##1}}}
\expandafter\def\csname PY@tok@sr\endcsname{\def\PY@tc##1{\textcolor[rgb]{0.73,0.40,0.53}{##1}}}
\expandafter\def\csname PY@tok@mo\endcsname{\def\PY@tc##1{\textcolor[rgb]{0.40,0.40,0.40}{##1}}}
\expandafter\def\csname PY@tok@kd\endcsname{\let\PY@bf=\textbf\def\PY@tc##1{\textcolor[rgb]{0.00,0.50,0.00}{##1}}}
\expandafter\def\csname PY@tok@mi\endcsname{\def\PY@tc##1{\textcolor[rgb]{0.40,0.40,0.40}{##1}}}
\expandafter\def\csname PY@tok@kn\endcsname{\let\PY@bf=\textbf\def\PY@tc##1{\textcolor[rgb]{0.00,0.50,0.00}{##1}}}
\expandafter\def\csname PY@tok@cpf\endcsname{\let\PY@it=\textit\def\PY@tc##1{\textcolor[rgb]{0.25,0.50,0.50}{##1}}}
\expandafter\def\csname PY@tok@kr\endcsname{\let\PY@bf=\textbf\def\PY@tc##1{\textcolor[rgb]{0.00,0.50,0.00}{##1}}}
\expandafter\def\csname PY@tok@s\endcsname{\def\PY@tc##1{\textcolor[rgb]{0.73,0.13,0.13}{##1}}}
\expandafter\def\csname PY@tok@kp\endcsname{\def\PY@tc##1{\textcolor[rgb]{0.00,0.50,0.00}{##1}}}
\expandafter\def\csname PY@tok@w\endcsname{\def\PY@tc##1{\textcolor[rgb]{0.73,0.73,0.73}{##1}}}
\expandafter\def\csname PY@tok@kt\endcsname{\def\PY@tc##1{\textcolor[rgb]{0.69,0.00,0.25}{##1}}}
\expandafter\def\csname PY@tok@sc\endcsname{\def\PY@tc##1{\textcolor[rgb]{0.73,0.13,0.13}{##1}}}
\expandafter\def\csname PY@tok@sb\endcsname{\def\PY@tc##1{\textcolor[rgb]{0.73,0.13,0.13}{##1}}}
\expandafter\def\csname PY@tok@sa\endcsname{\def\PY@tc##1{\textcolor[rgb]{0.73,0.13,0.13}{##1}}}
\expandafter\def\csname PY@tok@k\endcsname{\let\PY@bf=\textbf\def\PY@tc##1{\textcolor[rgb]{0.00,0.50,0.00}{##1}}}
\expandafter\def\csname PY@tok@se\endcsname{\let\PY@bf=\textbf\def\PY@tc##1{\textcolor[rgb]{0.73,0.40,0.13}{##1}}}
\expandafter\def\csname PY@tok@sd\endcsname{\let\PY@it=\textit\def\PY@tc##1{\textcolor[rgb]{0.73,0.13,0.13}{##1}}}

\def\PYZbs{\char`\\}
\def\PYZus{\char`\_}
\def\PYZob{\char`\{}
\def\PYZcb{\char`\}}
\def\PYZca{\char`\^}
\def\PYZam{\char`\&}
\def\PYZlt{\char`\<}
\def\PYZgt{\char`\>}
\def\PYZsh{\char`\#}
\def\PYZpc{\char`\%}
\def\PYZdl{\char`\$}
\def\PYZhy{\char`\-}
\def\PYZsq{\char`\'}
\def\PYZdq{\char`\"}
\def\PYZti{\char`\~}
% for compatibility with earlier versions
\def\PYZat{@}
\def\PYZlb{[}
\def\PYZrb{]}
\makeatother


    % Exact colors from NB
    \definecolor{incolor}{rgb}{0.0, 0.0, 0.5}
    \definecolor{outcolor}{rgb}{0.545, 0.0, 0.0}



    
    % Prevent overflowing lines due to hard-to-break entities
    \sloppy 
    % Setup hyperref package
    \hypersetup{
      breaklinks=true,  % so long urls are correctly broken across lines
      colorlinks=true,
      urlcolor=urlcolor,
      linkcolor=linkcolor,
      citecolor=citecolor,
      }
    % Slightly bigger margins than the latex defaults
    
    \geometry{verbose,tmargin=1in,bmargin=1in,lmargin=1in,rmargin=1in}
    
    

    \begin{document}
    
    
    \maketitle
    
    

    
    \section{Homework 5: Lexicons and Distributional
Semantics}\label{homework-5-lexicons-and-distributional-semantics}

This is due on \textbf{Friday, 11/10 (11pm)}

\subsection{How to do this problem
set}\label{how-to-do-this-problem-set}

Most of these questions require writing Python code and computing
results, and the rest of them have textual answers. Write all the
textual answers in this document, show the output of your experiment in
this document, and implement the functions in the \texttt{python} files.

Submit a PDF of thie .ipynb to Gradescope, and the .ipynb and all python
files to Moodle.

The assignment has two parts: * In the first part, you will experiment
with Turney's method to find word polarities in a twitter dataset, given
some positive and negative seed words. * In the second part, you will
experiment with distributional and vector semantics.

    \textbf{Your Name: Tapojit Debnath Tapu}

\textbf{List collaborators, and how you collaborated, here:} (see our
\href{http://people.cs.umass.edu/~brenocon/inlp2017/grading.html}{grading
and policies page} for details on our collaboration policy).

\begin{itemize}
\tightlist
\item
  \emph{name 1}
\end{itemize}

    \subsection{Part 1: Lexicon semantics}\label{part-1-lexicon-semantics}

    Recall that PMI of a pair of words, is defined as:

\[PMI(x, y) = log\frac{ P(x, y) }{ P(x)P(y)}\]

The Turney mehod defines a word's polarity as:

\[Polarity(word) = PMI(word, positive\_word)−PMI(word, negative\_word)\]

where the joint probability \(P(w, v)\) or, more specifically,
\(P(w\ NEAR\ v)\) is the probability of both being "near" each other.
We'll work with tweets, so it means: if you choose a tweet at random,
what's the chance it contains both \texttt{w} and \texttt{v}?

(If you look at the Turney method as explained in the SLP3 chapter, the
"hits" function is a count of web pages that contain at least one
instance of the two words occurring near each other.)

The positive\_word and negative\_word terms are initially constructed by
hand. For example: we might start with single positive word
('excellent') and a single negative word ('bad'). We can also have list
of positive words ('excellent', 'perfect', 'love', ....) and list of
negative words ('bad', 'hate', 'filthy',....)

If we're using a seed list of multiple terms, just combine them into a
single symbol, e.g. all the positive seed words get rewritten to
POS\_WORD (and similarly for NEG\_WORD). This \(P(w, POS\_WORD)\)
effectively means the co-ocurrence of \(w\) with any of the terms in the
list.

For this assignment, we will use twitter dataset which has 349378
tweets. These tweets are in the file named \texttt{tweets.txt}. These
are the tweets of one day and filtered such that each tweet contains at
least one of the seed words we've selected.

    \subsection{Question 1 (15 points)}\label{question-1-15-points}

The file \texttt{tweets.txt} contains around 349,378 tweets with one
tweet per line. It is a random sample of public tweets, which we
tokenized with
\href{https://github.com/myleott/ark-twokenize-py/blob/master/twokenize.py}{twokenize.py's
tokenizeRawTweetText()}). The text you see has a space between each
token so you can just use \texttt{.split()} if you want. We also
filtered tweets to ones that included at least one term from one of
these seed lists: * Positive seed list: {[}"good", "nice", "love",
"excellent", "fortunate", "correct", "superior"{]} * Negative seed list:
{[}"bad", "nasty", "poor", "hate", "unfortunate", "wrong", "inferior"{]}

Each tweet contains at least one positive or negative seed word. Take a
look at the file (e.g. \texttt{less\textquotesingle{}\ and}grep').
Implement the Turney's method to calculate polarity scores of all words.

Some things to keep in mind: * Ignore the seed words (i.e. don't
calculate the polarity of the seed words). * You may want to ignore the
polarity of words beignning with \texttt{@} or \texttt{\#}.

We recommend that you write code in a python file, but it's up to you.

QUESTION: You'll have to do something to handle zeros in the PMI
equation. Please explain your and justify your decision about this.

    \textbf{textual answer here}

    If P(x,y) in PMI equation is zero, it is assumed that PMI is zero. For
instance, for a given word, w, if P(w, POS\_WORD) is zero, then PMI for
it is zero. PMI for NEG\_WORD is asserted to be negative. IF P(w,
NEG\_WORD) is zero, PMI for it is zero. PMI for POS\_WORD is asserted to
be positive. If P(w, POS\_WORD) \& P(w, NEG\_WORD) are zero, then
polarity is zero. Zero polarity means word is neutral.

    \subsection{Question 2 (5 points)}\label{question-2-5-points}

Print the top 50 most-positive words (i.e. inferred positive words) and
the 50 most-negative words.

Many of the words won't make sense. Comment on at least two that do make
sense, and at least two that don't. Why do you think these are happening
with this dataset and method?

    \begin{Verbatim}[commandchars=\\\{\}]
{\color{incolor}In [{\color{incolor}1}]:} \PY{c+c1}{\PYZsh{} Write code to print words here}
        \PY{k+kn}{import} \PY{n+nn}{operator}
        \PY{k+kn}{from} \PY{n+nn}{collections} \PY{k+kn}{import} \PY{n}{defaultdict}
        \PY{k+kn}{from} \PY{n+nn}{polarity} \PY{k+kn}{import} \PY{n}{polarity\PYZus{}calc}
        \PY{n}{polarity}\PY{o}{=}\PY{n}{polarity\PYZus{}calc}\PY{p}{(}\PY{p}{)}
        \PY{n}{max\PYZus{}50}\PY{o}{=}\PY{n+nb}{dict}\PY{p}{(}\PY{n+nb}{sorted}\PY{p}{(}\PY{n}{polarity}\PY{o}{.}\PY{n}{iteritems}\PY{p}{(}\PY{p}{)}\PY{p}{,} \PY{n}{key}\PY{o}{=}\PY{n}{operator}\PY{o}{.}\PY{n}{itemgetter}\PY{p}{(}\PY{l+m+mi}{1}\PY{p}{)}\PY{p}{,} \PY{n}{reverse}\PY{o}{=}\PY{n+nb+bp}{True}\PY{p}{)}\PY{p}{[}\PY{p}{:}\PY{l+m+mi}{50}\PY{p}{]}\PY{p}{)}\PY{o}{.}\PY{n}{keys}\PY{p}{(}\PY{p}{)}
        \PY{n}{min\PYZus{}50}\PY{o}{=}\PY{n+nb}{dict}\PY{p}{(}\PY{n+nb}{sorted}\PY{p}{(}\PY{n}{polarity}\PY{o}{.}\PY{n}{iteritems}\PY{p}{(}\PY{p}{)}\PY{p}{,} \PY{n}{key}\PY{o}{=}\PY{n}{operator}\PY{o}{.}\PY{n}{itemgetter}\PY{p}{(}\PY{l+m+mi}{1}\PY{p}{)}\PY{p}{,} \PY{n}{reverse}\PY{o}{=}\PY{n+nb+bp}{True}\PY{p}{)}\PY{p}{[}\PY{o}{\PYZhy{}}\PY{l+m+mi}{50}\PY{p}{:}\PY{p}{]}\PY{p}{)}\PY{o}{.}\PY{n}{keys}\PY{p}{(}\PY{p}{)}
        \PY{k}{print} \PY{l+s+s2}{\PYZdq{}}\PY{l+s+s2}{50 most positive tokens: }\PY{l+s+s2}{\PYZdq{}}\PY{p}{,} \PY{l+s+s2}{\PYZdq{}}\PY{l+s+se}{\PYZbs{}n}\PY{l+s+s2}{\PYZdq{}}\PY{p}{,} \PY{n}{max\PYZus{}50}\PY{p}{,} \PY{l+s+s2}{\PYZdq{}}\PY{l+s+se}{\PYZbs{}n}\PY{l+s+s2}{\PYZdq{}}
        \PY{k}{print} \PY{l+s+s2}{\PYZdq{}}\PY{l+s+s2}{50 most negative tokens: }\PY{l+s+s2}{\PYZdq{}}\PY{p}{,} \PY{l+s+s2}{\PYZdq{}}\PY{l+s+se}{\PYZbs{}n}\PY{l+s+s2}{\PYZdq{}}\PY{p}{,} \PY{n}{min\PYZus{}50}
\end{Verbatim}


    \begin{Verbatim}[commandchars=\\\{\}]
50 most positive tokens:  
['friend\textbackslash{}xf0\textbackslash{}x9f\textbackslash{}xa4\textbackslash{}x9e\textbackslash{}xf0\textbackslash{}x9f\textbackslash{}x8f\textbackslash{}xbd', 'https://t.co/unteosecly', 'cassievers\textbackslash{}xe2\textbackslash{}x98\textbackslash{}xba', 'https://t.co/rcgdsyymir', 'eighteet', 'you\textbackslash{}\textbackslash{}n-i', '-*', 'througho\textbackslash{}xe2\textbackslash{}x80\textbackslash{}xa6', 'one\textbackslash{}\textbackslash{}ngood', 'gf\textbackslash{}xf0\textbackslash{}x9f\textbackslash{}x92\textbackslash{}x8d', 'ex\textbackslash{}xf0\textbackslash{}x9f\textbackslash{}x97\textbackslash{}x91', 'amiracan', 'lovvveeeee', 'https://t.co/filzv6f2ze', 'dulce\textbackslash{}xf0\textbackslash{}x9f\textbackslash{}x98\textbackslash{}xa9', 'https://t.co/lvrki7uwhd', 'you\textbackslash{}xc2\textbackslash{}xa1\textbackslash{}xc2\textbackslash{}xa1\textbackslash{}\textbackslash{}', 'chazelle', 'n-i', 'dirty\textbackslash{}xf0\textbackslash{}x9f\textbackslash{}xa4\textbackslash{}xa2', 'aside\textbackslash{}\textbackslash{}', 'it\textbackslash{}xf0\textbackslash{}x9f\textbackslash{}xa4\textbackslash{}x91', 'https://t.co/p1fyayehnz', '-tb\textbackslash{}\textbackslash{}', 'https://t.co/xketcvmmno', 'https://t.co/4vy3fv70rk', 'https://t.co/38zyfmjco8', '\textbackslash{}xcb\textbackslash{}x97\textbackslash{}xcb\textbackslash{}x8f\textbackslash{}xcb\textbackslash{}x8b', 'https://t.co/pyvhn56pje', 'https://t.co/ve99xlwp0g', 'https://t.co/bnk2hia1mk', 'nori\textbackslash{}xe2\textbackslash{}x80\textbackslash{}xa6', 'forever\textbackslash{}\textbackslash{}n-i', 'todos', 'much\textbackslash{}\textbackslash{}n-i', 'noriega', 'bidadari', 'https://t.co/o1ii42byhl', 'bodoamatlah', '\textbackslash{}xcb\textbackslash{}x8e\textbackslash{}xcb\textbackslash{}x8a\textbackslash{}xcb\textbackslash{}x97', 'girl\textbackslash{}xf0\textbackslash{}x9f\textbackslash{}x98\textbackslash{}xad\textbackslash{}xf0\textbackslash{}x9f\textbackslash{}x98\textbackslash{}xad\textbackslash{}xf0\textbackslash{}x9f\textbackslash{}x98\textbackslash{}xadits', 'luck\textbackslash{}xf0\textbackslash{}x9f\textbackslash{}x92\textbackslash{}x9e\textbackslash{}xf0\textbackslash{}x9f\textbackslash{}x92\textbackslash{}x9c', '\textbackslash{}xe2\textbackslash{}x9d\textbackslash{}xa4\textbackslash{}xf0\textbackslash{}x9f\textbackslash{}x92\textbackslash{}x99\textbackslash{}xf0\textbackslash{}x9f\textbackslash{}x92\textbackslash{}x9b\textbackslash{}xf0\textbackslash{}x9f\textbackslash{}x92\textbackslash{}x9c', '2017.02', 'https://t.co/hsyskqcnn6', 'octavia', '\textbackslash{}xe2\textbackslash{}x9b\textbackslash{}x88\textbackslash{}xe2\textbackslash{}x9a\textbackslash{}xa1\textbackslash{}xef\textbackslash{}xb8\textbackslash{}x8f\textbackslash{}xf0\textbackslash{}x9f\textbackslash{}x98\textbackslash{}x8d', '\textbackslash{}\textbackslash{}n\textbackslash{}\textbackslash{}nfaveeeee', 'looooooovveeee', 'https://t.co/jm\textbackslash{}xe2\textbackslash{}x80\textbackslash{}xa6'] 

50 most negative tokens:  
['us-they', 'sudmalis', 'bitchness', 'deomocrats', 'https://t.co/tx\textbackslash{}xe2\textbackslash{}x80\textbackslash{}xa6', 'https://t.co/fxveuydvfx', 'hurt/disregarded', 'https://t.co/uchmaxafrk', 'looser', 'https://t.co/cmz27nmopl', "day's.", 'https://t.co/rjmh9eh\textbackslash{}xe2\textbackslash{}x80\textbackslash{}xa6', 'all-i', 'usa-they', 'https://t.co/rnqhff13az', 'https://t.co/mtehygjxtl', 'democracy\textbackslash{}\textbackslash{}nleftists', 'look\textbackslash{}xf0\textbackslash{}x9f\textbackslash{}x98\textbackslash{}x85', 'wowowowowowowowow', 'winnn', 'ill-they', 'anti-patriotism', 'bumbling', 'https://t.co/qio17l0iqz', 'proclivity', 'chavs', 'congressperson-to-congressperson', 'https://t.co/2tiza9n7\textbackslash{}xe2\textbackslash{}x80\textbackslash{}xa6', 'nap/bad', '\textbackslash{}xf0\textbackslash{}x9f\textbackslash{}x93\textbackslash{}x8csaying', 'https://t.co/wwfvgyfolm', 'racist\textbackslash{}\textbackslash{}', '4cm', 'my-', 'home\textbackslash{}xf0\textbackslash{}x9f\textbackslash{}x98\textbackslash{}xad', 'l\textbackslash{}xc3\textbackslash{}xb8v\textbackslash{}xc3\textbackslash{}xab', 'https://t.co/lqwhh2qora', 'torbjorn', 'hinduism\textbackslash{}\textbackslash{}nyet', 'nazi\textbackslash{}\textbackslash{}', 'india\textbackslash{}\textbackslash{}nleftists', 'cough*cincy*cough', 'https://t.co/wzz4ndbn7n', '||i', 'https://t.co/qfpq8hcd3l', 'https://t.co/tkzqcrnry8', 'cialis', 'condiment', 'https://t.co/yjikxov0q9', 'https://t.co/itqwxo\textbackslash{}xe2\textbackslash{}x80\textbackslash{}xa6']

    \end{Verbatim}

    \subsubsection{Textual answer here.}\label{textual-answer-here.}

    Two that make sense: 'looooooovveeee' for positive tokens, since it is
short for \texttt{love} and 'looser' from negative tokens, which means
\texttt{loser}. Two that do not make sense: 'octavia' for positive
tokens and 'condiment' from negative tokens. Both are supposed to be
neutral words. They do not make sense since word in a tweet is
considered to be positive or negative if there is a positive or negative
seed word in said tweet respectively.

    \subsection{Question 3 (5 points)}\label{question-3-5-points}

Now filter out all the words which have total count \textless{} 500, and
then print top 50 polarity words and bottom 50 polarity words.

Choose some of the words from both the sets of 50 words you got above
which accoording to you make sense. Again please note, you will find
many words which don't make sense. Do you think these results are better
than the results you got in Question-1? Explain why.

    \begin{Verbatim}[commandchars=\\\{\}]
{\color{incolor}In [{\color{incolor}2}]:} \PY{c+c1}{\PYZsh{} Write code to print words here}
        \PY{k+kn}{import} \PY{n+nn}{operator}
        \PY{k+kn}{from} \PY{n+nn}{collections} \PY{k+kn}{import} \PY{n}{defaultdict}
        \PY{k+kn}{from} \PY{n+nn}{polarity} \PY{k+kn}{import} \PY{n}{polarity\PYZus{}calc}
        \PY{n}{polarity}\PY{o}{=}\PY{n}{polarity\PYZus{}calc}\PY{p}{(}\PY{n}{filter\PYZus{}500}\PY{o}{=}\PY{n+nb+bp}{True}\PY{p}{)}
        \PY{n}{max\PYZus{}50}\PY{o}{=}\PY{n+nb}{dict}\PY{p}{(}\PY{n+nb}{sorted}\PY{p}{(}\PY{n}{polarity}\PY{o}{.}\PY{n}{iteritems}\PY{p}{(}\PY{p}{)}\PY{p}{,} \PY{n}{key}\PY{o}{=}\PY{n}{operator}\PY{o}{.}\PY{n}{itemgetter}\PY{p}{(}\PY{l+m+mi}{1}\PY{p}{)}\PY{p}{,} \PY{n}{reverse}\PY{o}{=}\PY{n+nb+bp}{True}\PY{p}{)}\PY{p}{[}\PY{p}{:}\PY{l+m+mi}{50}\PY{p}{]}\PY{p}{)}\PY{o}{.}\PY{n}{keys}\PY{p}{(}\PY{p}{)}
        \PY{n}{min\PYZus{}50}\PY{o}{=}\PY{n+nb}{dict}\PY{p}{(}\PY{n+nb}{sorted}\PY{p}{(}\PY{n}{polarity}\PY{o}{.}\PY{n}{iteritems}\PY{p}{(}\PY{p}{)}\PY{p}{,} \PY{n}{key}\PY{o}{=}\PY{n}{operator}\PY{o}{.}\PY{n}{itemgetter}\PY{p}{(}\PY{l+m+mi}{1}\PY{p}{)}\PY{p}{,} \PY{n}{reverse}\PY{o}{=}\PY{n+nb+bp}{True}\PY{p}{)}\PY{p}{[}\PY{o}{\PYZhy{}}\PY{l+m+mi}{50}\PY{p}{:}\PY{p}{]}\PY{p}{)}\PY{o}{.}\PY{n}{keys}\PY{p}{(}\PY{p}{)}
        \PY{k}{print} \PY{l+s+s2}{\PYZdq{}}\PY{l+s+s2}{50 most positive tokens: }\PY{l+s+s2}{\PYZdq{}}\PY{p}{,} \PY{l+s+s2}{\PYZdq{}}\PY{l+s+se}{\PYZbs{}n}\PY{l+s+s2}{\PYZdq{}}\PY{p}{,} \PY{n}{max\PYZus{}50}\PY{p}{,} \PY{l+s+s2}{\PYZdq{}}\PY{l+s+se}{\PYZbs{}n}\PY{l+s+s2}{\PYZdq{}}
        \PY{k}{print} \PY{l+s+s2}{\PYZdq{}}\PY{l+s+s2}{50 most negative tokens: }\PY{l+s+s2}{\PYZdq{}}\PY{p}{,} \PY{l+s+s2}{\PYZdq{}}\PY{l+s+se}{\PYZbs{}n}\PY{l+s+s2}{\PYZdq{}}\PY{p}{,} \PY{n}{min\PYZus{}50}
\end{Verbatim}


    \begin{Verbatim}[commandchars=\\\{\}]
50 most positive tokens:  
['friend\textbackslash{}xf0\textbackslash{}x9f\textbackslash{}xa4\textbackslash{}x9e\textbackslash{}xf0\textbackslash{}x9f\textbackslash{}x8f\textbackslash{}xbd', 'https://t.co/unteosecly', 'cassievers\textbackslash{}xe2\textbackslash{}x98\textbackslash{}xba', 'https://t.co/rcgdsyymir', 'eighteet', 'you\textbackslash{}\textbackslash{}n-i', '-*', 'througho\textbackslash{}xe2\textbackslash{}x80\textbackslash{}xa6', 'one\textbackslash{}\textbackslash{}ngood', 'gf\textbackslash{}xf0\textbackslash{}x9f\textbackslash{}x92\textbackslash{}x8d', 'ex\textbackslash{}xf0\textbackslash{}x9f\textbackslash{}x97\textbackslash{}x91', 'amiracan', 'lovvveeeee', 'https://t.co/filzv6f2ze', 'dulce\textbackslash{}xf0\textbackslash{}x9f\textbackslash{}x98\textbackslash{}xa9', 'https://t.co/lvrki7uwhd', 'you\textbackslash{}xc2\textbackslash{}xa1\textbackslash{}xc2\textbackslash{}xa1\textbackslash{}\textbackslash{}', 'chazelle', 'n-i', 'dirty\textbackslash{}xf0\textbackslash{}x9f\textbackslash{}xa4\textbackslash{}xa2', 'aside\textbackslash{}\textbackslash{}', 'it\textbackslash{}xf0\textbackslash{}x9f\textbackslash{}xa4\textbackslash{}x91', 'https://t.co/p1fyayehnz', '-tb\textbackslash{}\textbackslash{}', 'https://t.co/xketcvmmno', 'https://t.co/4vy3fv70rk', 'https://t.co/38zyfmjco8', '\textbackslash{}xcb\textbackslash{}x97\textbackslash{}xcb\textbackslash{}x8f\textbackslash{}xcb\textbackslash{}x8b', 'https://t.co/pyvhn56pje', 'https://t.co/ve99xlwp0g', 'https://t.co/bnk2hia1mk', 'nori\textbackslash{}xe2\textbackslash{}x80\textbackslash{}xa6', 'forever\textbackslash{}\textbackslash{}n-i', 'todos', 'much\textbackslash{}\textbackslash{}n-i', 'noriega', 'bidadari', 'https://t.co/o1ii42byhl', 'bodoamatlah', '\textbackslash{}xcb\textbackslash{}x8e\textbackslash{}xcb\textbackslash{}x8a\textbackslash{}xcb\textbackslash{}x97', 'girl\textbackslash{}xf0\textbackslash{}x9f\textbackslash{}x98\textbackslash{}xad\textbackslash{}xf0\textbackslash{}x9f\textbackslash{}x98\textbackslash{}xad\textbackslash{}xf0\textbackslash{}x9f\textbackslash{}x98\textbackslash{}xadits', 'luck\textbackslash{}xf0\textbackslash{}x9f\textbackslash{}x92\textbackslash{}x9e\textbackslash{}xf0\textbackslash{}x9f\textbackslash{}x92\textbackslash{}x9c', '\textbackslash{}xe2\textbackslash{}x9d\textbackslash{}xa4\textbackslash{}xf0\textbackslash{}x9f\textbackslash{}x92\textbackslash{}x99\textbackslash{}xf0\textbackslash{}x9f\textbackslash{}x92\textbackslash{}x9b\textbackslash{}xf0\textbackslash{}x9f\textbackslash{}x92\textbackslash{}x9c', '2017.02', 'https://t.co/hsyskqcnn6', 'octavia', '\textbackslash{}xe2\textbackslash{}x9b\textbackslash{}x88\textbackslash{}xe2\textbackslash{}x9a\textbackslash{}xa1\textbackslash{}xef\textbackslash{}xb8\textbackslash{}x8f\textbackslash{}xf0\textbackslash{}x9f\textbackslash{}x98\textbackslash{}x8d', '\textbackslash{}\textbackslash{}n\textbackslash{}\textbackslash{}nfaveeeee', 'looooooovveeee', 'https://t.co/jm\textbackslash{}xe2\textbackslash{}x80\textbackslash{}xa6'] 

50 most negative tokens:  
['us-they', 'sudmalis', 'bitchness', 'deomocrats', 'https://t.co/tx\textbackslash{}xe2\textbackslash{}x80\textbackslash{}xa6', 'https://t.co/fxveuydvfx', 'hurt/disregarded', 'https://t.co/uchmaxafrk', 'looser', 'https://t.co/cmz27nmopl', "day's.", 'https://t.co/rjmh9eh\textbackslash{}xe2\textbackslash{}x80\textbackslash{}xa6', 'all-i', 'usa-they', 'https://t.co/rnqhff13az', 'https://t.co/mtehygjxtl', 'democracy\textbackslash{}\textbackslash{}nleftists', 'look\textbackslash{}xf0\textbackslash{}x9f\textbackslash{}x98\textbackslash{}x85', 'wowowowowowowowow', 'winnn', 'ill-they', 'anti-patriotism', 'bumbling', 'https://t.co/qio17l0iqz', 'proclivity', 'chavs', 'congressperson-to-congressperson', 'https://t.co/2tiza9n7\textbackslash{}xe2\textbackslash{}x80\textbackslash{}xa6', 'nap/bad', '\textbackslash{}xf0\textbackslash{}x9f\textbackslash{}x93\textbackslash{}x8csaying', 'https://t.co/wwfvgyfolm', 'racist\textbackslash{}\textbackslash{}', '4cm', 'my-', 'home\textbackslash{}xf0\textbackslash{}x9f\textbackslash{}x98\textbackslash{}xad', 'l\textbackslash{}xc3\textbackslash{}xb8v\textbackslash{}xc3\textbackslash{}xab', 'https://t.co/lqwhh2qora', 'torbjorn', 'hinduism\textbackslash{}\textbackslash{}nyet', 'nazi\textbackslash{}\textbackslash{}', 'india\textbackslash{}\textbackslash{}nleftists', 'cough*cincy*cough', 'https://t.co/wzz4ndbn7n', '||i', 'https://t.co/qfpq8hcd3l', 'https://t.co/tkzqcrnry8', 'cialis', 'condiment', 'https://t.co/yjikxov0q9', 'https://t.co/itqwxo\textbackslash{}xe2\textbackslash{}x80\textbackslash{}xa6']

    \end{Verbatim}

    \subsubsection{Textual answer here.}\label{textual-answer-here.}

    Words which make sense:
\textbf{'friend\xf0\x9f\xa4\x9e\xf0\x9f\x8f\xbd',
'luck\xf0\x9f\x92\x9e\xf0\x9f\x92\x9c'} from positive tokens(Only their
beginning parts like 'friend' and 'luck' make
sense).\textless{}br.\textgreater{} words which do not make sense:
\textbf{'wowowowowowowowow', 'winnn'} from negative tokens, since they
are supposed to be in positive token group. Results have not improved at
all; corresponding token sets from question 2 and question 3 are still
strikingly similar. This may be because most tokens have frequency below
500. Owing to their low frequency, they are most likely to be associated
with either positive or negative group, even if high frequency words are
filtered out.

    \subsection{Question 4 (5 points)}\label{question-4-5-points}

Even after filtering out words with count \textless{} 500, many top-most
and bottom-most polarity don't make sense. Identify what kind of words
these are and what can be done to filter them out. You can read some
tweets in the file to see what's happening.

    \subsubsection{Textual answer here.}\label{textual-answer-here.}

    These "tokens" mostly consist of emojis, like
\texttt{friend\textbackslash{}xf0\textbackslash{}x9f\textbackslash{}xa4\textbackslash{}x9e\textbackslash{}xf0\textbackslash{}x9f\textbackslash{}x8f\textbackslash{}xbd}.
The 'friend' part is distinguishable, but characters after it represent
emojis. Moreover, there are a lot of html links too, like
\texttt{https://t.co/jm\textbackslash{}xe2\textbackslash{}x80\textbackslash{}xa6},
which should be removed. HTML links can be identified by checking
whether the first 4 characters of a token are equivalent to the string
'http'. If fourth character from end of token string is a backslash
\texttt{\textbackslash{}}, then it means this particular token consist
mostly of emojis and can be removed.

    \section{Part-2: Distributional
Semantics}\label{part-2-distributional-semantics}

    \subsection{Cosine Similarity}\label{cosine-similarity}

    Recall that, where \(i\) indexes over the context types, cosine
similarity is defined as follows. \(x\) and \(y\) are both vectors of
context counts (each for a different word), where \(x_i\) is the count
of context \(i\).

\[cossim(x,y) = \frac{ \sum_i x_i y_i }{ \sqrt{\sum_i x_i^2} \sqrt{\sum_i y_i^2} }\]

The nice thing about cosine similarity is that it is normalized: no
matter what the input vectors are, the output is between 0 and 1. One
way to think of this is that cosine similarity is just, um, the cosine
function, which has this property (for non-negative \(x\) and \(y\)).
Another way to think of it is, to work through the situations of maximum
and minimum similarity between two context vectors, starting from the
definition above.

Note: a good way to understand the cosine similarity function is that
the numerator cares about whether the \(x\) and \(y\) vectors are
correlated. If \(x\) and \(y\) tend to have high values for the same
contexts, the numerator tends to be big. The denominator can be thought
of as a normalization factor: if all the values of \(x\) are really
large, for example, dividing by the square root of their sum-of-squares
prevents the whole thing from getting arbitrarily large. In fact,
dividing by both these things (aka their norms) means the whole thing
can't go higher than 1.

In this problem we'll work with vectors of raw context counts. (As you
know, this is not necessarily the best representation.)

    \subsection{Question 5 (5 points)}\label{question-5-5-points}

See the file \texttt{nytcounts.university\_cat\_dog}, which contains
context count vectors for three words: ``dog'', ``cat'', and
``university''. These are immediate left and right contexts from a New
York Times corpus. You can open the file in a text editor since it's
quite small.

Write a function which takes context count dictionaries of two words and
calculates cosine similarity between these two words. The function
should return a number beween 0 and 1. Briefly comment on whether the
relative simlarities make sense.

    \begin{Verbatim}[commandchars=\\\{\}]
{\color{incolor}In [{\color{incolor}1}]:} \PY{k+kn}{import} \PY{n+nn}{distsim}\PY{p}{;} \PY{n+nb}{reload}\PY{p}{(}\PY{n}{distsim}\PY{p}{)}
        
        \PY{n}{word\PYZus{}to\PYZus{}ccdict} \PY{o}{=} \PY{n}{distsim}\PY{o}{.}\PY{n}{load\PYZus{}contexts}\PY{p}{(}\PY{l+s+s2}{\PYZdq{}}\PY{l+s+s2}{nytcounts.university\PYZus{}cat\PYZus{}dog}\PY{l+s+s2}{\PYZdq{}}\PY{p}{)}
        \PY{n}{cat\PYZus{}dog\PYZus{}sim}\PY{o}{=}\PY{n}{distsim}\PY{o}{.}\PY{n}{cos\PYZus{}sim}\PY{p}{(}\PY{n}{word\PYZus{}to\PYZus{}ccdict}\PY{p}{,}\PY{l+s+s1}{\PYZsq{}}\PY{l+s+s1}{cat}\PY{l+s+s1}{\PYZsq{}}\PY{p}{,}\PY{l+s+s1}{\PYZsq{}}\PY{l+s+s1}{dog}\PY{l+s+s1}{\PYZsq{}}\PY{p}{)}
        \PY{n}{cat\PYZus{}uni\PYZus{}sim}\PY{o}{=}\PY{n}{distsim}\PY{o}{.}\PY{n}{cos\PYZus{}sim}\PY{p}{(}\PY{n}{word\PYZus{}to\PYZus{}ccdict}\PY{p}{,}\PY{l+s+s1}{\PYZsq{}}\PY{l+s+s1}{cat}\PY{l+s+s1}{\PYZsq{}}\PY{p}{,}\PY{l+s+s1}{\PYZsq{}}\PY{l+s+s1}{university}\PY{l+s+s1}{\PYZsq{}}\PY{p}{)}
        \PY{n}{dog\PYZus{}uni\PYZus{}sim}\PY{o}{=}\PY{n}{distsim}\PY{o}{.}\PY{n}{cos\PYZus{}sim}\PY{p}{(}\PY{n}{word\PYZus{}to\PYZus{}ccdict}\PY{p}{,}\PY{l+s+s1}{\PYZsq{}}\PY{l+s+s1}{university}\PY{l+s+s1}{\PYZsq{}}\PY{p}{,}\PY{l+s+s1}{\PYZsq{}}\PY{l+s+s1}{dog}\PY{l+s+s1}{\PYZsq{}}\PY{p}{)}
        
        \PY{c+c1}{\PYZsh{} write code here to show output (i.e. cosine similarity between these words.)}
        \PY{c+c1}{\PYZsh{} We encourage you to write other functions in distsim.py itself.}
        \PY{k}{print} \PY{l+s+s2}{\PYZdq{}}\PY{l+s+se}{\PYZbs{}n}\PY{l+s+s2}{cosine similarity between cat and dog: }\PY{l+s+s2}{\PYZdq{}}\PY{p}{,} \PY{n}{cat\PYZus{}dog\PYZus{}sim}\PY{p}{,} \PY{l+s+s2}{\PYZdq{}}\PY{l+s+se}{\PYZbs{}n}\PY{l+s+s2}{\PYZdq{}}
        \PY{k}{print} \PY{l+s+s2}{\PYZdq{}}\PY{l+s+s2}{cosine similarity between cat and university: }\PY{l+s+s2}{\PYZdq{}}\PY{p}{,} \PY{n}{cat\PYZus{}uni\PYZus{}sim}\PY{p}{,} \PY{l+s+s2}{\PYZdq{}}\PY{l+s+se}{\PYZbs{}n}\PY{l+s+s2}{\PYZdq{}}
        \PY{k}{print} \PY{l+s+s2}{\PYZdq{}}\PY{l+s+s2}{cosine similarity between university and dog: }\PY{l+s+s2}{\PYZdq{}}\PY{p}{,} \PY{n}{dog\PYZus{}uni\PYZus{}sim}
\end{Verbatim}


    \begin{Verbatim}[commandchars=\\\{\}]
file nytcounts.university\_cat\_dog has contexts for 3 words

cosine similarity between cat and dog:  0.966891672715 

cosine similarity between cat and university:  0.660442421144 

cosine similarity between university and dog:  0.659230248969

    \end{Verbatim}

    \textbf{Write your response here:}

    They do make sense, since cosine similarity between \texttt{cat} \&
\texttt{dog} is highest(both are animals) in comparison with that
between \texttt{cat} \& \texttt{university} and \texttt{dog} \&
\texttt{university}.

    \subsection{Question 6 (20 points)}\label{question-6-20-points}

Explore similarities in \texttt{nytcounts.4k}, which contains context
counts for about 4000 words in a sample of New York Times articles. The
news data was lowercased and URLs were removed. The context counts are
for the 2000 most common words in twitter, as well as the most common
2000 words in the New York Times. (But all context counts are from New
York Times.) The context counts only contain contexts that appeared for
more than one word. The file has three tab-separate fields: the word,
its count, and a JSON-encoded dictionary of its context counts. You'll
see it's just counts of the immediate left/right neighbors.

Choose \textbf{six} words. For each, show the output of 20 nearest words
(use cosine similarity as distance metric). Comment on whether the
output makes sense. Comment on whether this approach to distributional
similarity makes more or less sense for certain terms. Four of your
words should be:

\begin{itemize}
\tightlist
\item
  a name (for example: person, organization, or location)
\item
  a common noun
\item
  an adjective
\item
  a verb
\end{itemize}

You may also want to try exploring further words that are returned from
a most-similar list from one of these. You can think of this as
traversing the similarity graph among words.

\emph{Implementation note:} On my laptop it takes several hundred MB of
memory to load it into memory from the \texttt{load\_contexts()}
function. If you don't have enough memory available, your computer will
get very slow because the OS will start swapping. If you have to use a
machine without that much memory available, you can instead implement in
a streaming approach by using the \texttt{stream\_contexts()} generator
function to access the data; this lets you iterate through the data from
disk, one vector at a time, without putting everything into memory. You
can see its use in the loading function. (You could also alternatively
use a key-value or other type of database, but that's too much work for
this assignment.)

    \begin{Verbatim}[commandchars=\\\{\}]
{\color{incolor}In [{\color{incolor}1}]:} \PY{k+kn}{import} \PY{n+nn}{distsim}\PY{p}{;} \PY{n+nb}{reload}\PY{p}{(}\PY{n}{distsim}\PY{p}{)}
        \PY{k+kn}{import} \PY{n+nn}{operator}
        \PY{n}{word\PYZus{}list}\PY{o}{=}\PY{p}{[}\PY{l+s+s2}{\PYZdq{}}\PY{l+s+s2}{stock}\PY{l+s+s2}{\PYZdq{}}\PY{p}{,} \PY{l+s+s2}{\PYZdq{}}\PY{l+s+s2}{limited}\PY{l+s+s2}{\PYZdq{}}\PY{p}{,} \PY{l+s+s2}{\PYZdq{}}\PY{l+s+s2}{inning}\PY{l+s+s2}{\PYZdq{}}\PY{p}{,} \PY{l+s+s2}{\PYZdq{}}\PY{l+s+s2}{yellow}\PY{l+s+s2}{\PYZdq{}}\PY{p}{,} \PY{l+s+s2}{\PYZdq{}}\PY{l+s+s2}{saved}\PY{l+s+s2}{\PYZdq{}}\PY{p}{,} \PY{l+s+s2}{\PYZdq{}}\PY{l+s+s2}{paris}\PY{l+s+s2}{\PYZdq{}}\PY{p}{]}
        
        \PY{k}{for} \PY{n}{word} \PY{o+ow}{in} \PY{n}{word\PYZus{}list}\PY{p}{:}
            \PY{n}{dict\PYZus{}cos\PYZus{}sim}\PY{o}{=}\PY{n}{distsim}\PY{o}{.}\PY{n}{stream\PYZus{}cos\PYZus{}sim\PYZus{}calc}\PY{p}{(}\PY{n}{word}\PY{p}{)}
            \PY{n}{near\PYZus{}20}\PY{o}{=}\PY{n+nb}{dict}\PY{p}{(}\PY{n+nb}{sorted}\PY{p}{(}\PY{n}{dict\PYZus{}cos\PYZus{}sim}\PY{o}{.}\PY{n}{iteritems}\PY{p}{(}\PY{p}{)}\PY{p}{,} \PY{n}{key}\PY{o}{=}\PY{n}{operator}\PY{o}{.}\PY{n}{itemgetter}\PY{p}{(}\PY{l+m+mi}{1}\PY{p}{)}\PY{p}{,} \PY{n}{reverse}\PY{o}{=}\PY{n+nb+bp}{True}\PY{p}{)}\PY{p}{[}\PY{p}{:}\PY{l+m+mi}{20}\PY{p}{]}\PY{p}{)}\PY{o}{.}\PY{n}{keys}\PY{p}{(}\PY{p}{)}
            
            \PY{k}{print} \PY{l+s+s2}{\PYZdq{}}\PY{l+s+se}{\PYZbs{}n}\PY{l+s+s2}{20 words nearest to }\PY{l+s+s2}{\PYZdq{}}\PY{p}{,} \PY{n}{word}\PY{p}{,} \PY{l+s+s2}{\PYZdq{}}\PY{l+s+s2}{:}\PY{l+s+se}{\PYZbs{}n}\PY{l+s+s2}{ }\PY{l+s+s2}{\PYZdq{}}\PY{p}{,} \PY{n}{near\PYZus{}20}\PY{p}{,} \PY{l+s+s2}{\PYZdq{}}\PY{l+s+se}{\PYZbs{}n}\PY{l+s+s2}{\PYZdq{}}
        
        \PY{c+c1}{\PYZsh{}\PYZsh{}\PYZsh{}Provide your answer below; perhaps in another cell so you don\PYZsq{}t have to reload the data each time}
\end{Verbatim}


    \begin{Verbatim}[commandchars=\\\{\}]

20 words nearest to  stock :
  ['state', 'bathroom', 'ball', 'mood', 'government', 'sun', 'gym', 'title', 'overall', 'primary', 'roof', 'budget', 'brain', 'game', 'future', 'shooting', 'economy', 'heat', 'film', 'bond'] 


20 words nearest to  limited :
  ['desire', 'closer', 'required', 'wise', 'continuing', 'hard', 'crucial', 'loyal', 'related', 'guide', 'necessary', 'chance', 'plan', 'quick', 'close', 'struggling', 'difficult', 'commitment', 'trip', 'happy'] 


20 words nearest to  inning :
  ['worldwide', 'partners', 'lane', 'session', 'floor', 'grade', 'cemetery', 'term', 'garage', 'street', 'systems', 'base', 'time', 'films', 'texts', 'foundation', 'quarter', 'avenue', 'road', 'movement'] 


20 words nearest to  yellow :
  ['pink', 'hip', 'fashion', 'sweet', 'green', 'gorgeous', 'cheap', 'quiet', 'fat', 'dry', 'bright', 'warm', 'black', 'dirty', 'rich', 'fresh', 'sexy', 'chicken', 'soft', 'pizza'] 


20 words nearest to  saved :
  ['raised', 'made', 'kept', 'missed', 'lost', 'left', 'given', 'changed', 'joined', 'rejected', 'visited', 'supported', 'pushed', 'won', 'stopped', 'watched', 'reached', 'ordered', 'accepted', 'blocked'] 


20 words nearest to  paris :
  ['europe', 'afghanistan', '1995', '1994', '1997', '1996', '1999', '1998', 'washington', 'baghdad', '2002', '2003', '2000', '2001', 'london', 'september', 'jail', 'atlanta', 'iraq', 'manhattan'] 


    \end{Verbatim}

    \textbf{Textual response} For words in the word list like \emph{yellow},
\emph{paris}, respective 20 nearest words obtained for each of them make
sense. For instance, for \emph{yellow}, many words closest to it are
also colors. For \emph{paris}, many words closest to it are names of
places and cities. But for \emph{stock}, \emph{limited}, \emph{saved}
and \emph{inning}, results are more sparse. But still, cosine similarity
algorithm picked up some words very close to their respective concepts.
In case of \emph{stock}, words like \emph{budget, economy, brain,
government} were closest to it. For terms with very few words in same
category as themselves (like \emph{stock}), the cosine similarity
algorithm does not perform so well. But in case of adjectives and nouns
which fall within categories like color, places, the algorithm performs
very well.

    \subsection{Question 7 (10 points)}\label{question-7-10-points}

In the next several questions, you'll examine similarities in trained
word embeddings, instead of raw context counts.

See the file \texttt{nyt\_word2vec.university\_cat\_dog}, which contains
word embedding vectors pretrained by word2vec {[}1{]} for three words:
``dog'', ``cat'', and ``university'', from the same corpus. You can open
the file in a text editor since it's quite small.

Write a function which takes word embedding vectors of two words and
calculates cossine similarity between these 2 words. The function should
return a number beween -1 and 1. Briefly comment on whether the relative
simlarities make sense.

\emph{Implementation note:} Notice that the inputs of this function are
numpy arrays (v1 and v2). If you are not very familiar with the basic
operation in numpy, you can find some examples in the basic operation
section here: https://docs.scipy.org/doc/numpy-dev/user/quickstart.html

If you know how to use Matlab but haven't tried numpy before, the
following link should be helpful:
https://docs.scipy.org/doc/numpy-dev/user/numpy-for-matlab-users.html

{[}1{]} Mikolov, Tomas, et al. "Distributed representations of words and
phrases and their compositionality." NIPS 2013.

    \begin{Verbatim}[commandchars=\\\{\}]
{\color{incolor}In [{\color{incolor}1}]:} \PY{k+kn}{import} \PY{n+nn}{distsim}\PY{p}{;} \PY{n+nb}{reload}\PY{p}{(}\PY{n}{distsim}\PY{p}{)}
        
        \PY{n}{word\PYZus{}to\PYZus{}vec\PYZus{}dict} \PY{o}{=} \PY{n}{distsim}\PY{o}{.}\PY{n}{load\PYZus{}word2vec}\PY{p}{(}\PY{l+s+s2}{\PYZdq{}}\PY{l+s+s2}{nyt\PYZus{}word2vec.university\PYZus{}cat\PYZus{}dog}\PY{l+s+s2}{\PYZdq{}}\PY{p}{)}
        
        
        \PY{n}{cat\PYZus{}dog\PYZus{}sim}\PY{o}{=}\PY{n}{distsim}\PY{o}{.}\PY{n}{cos\PYZus{}sim\PYZus{}word2vec}\PY{p}{(}\PY{n}{word\PYZus{}to\PYZus{}vec\PYZus{}dict}\PY{p}{,}\PY{l+s+s1}{\PYZsq{}}\PY{l+s+s1}{cat}\PY{l+s+s1}{\PYZsq{}}\PY{p}{,}\PY{l+s+s1}{\PYZsq{}}\PY{l+s+s1}{dog}\PY{l+s+s1}{\PYZsq{}}\PY{p}{)}
        \PY{n}{cat\PYZus{}uni\PYZus{}sim}\PY{o}{=}\PY{n}{distsim}\PY{o}{.}\PY{n}{cos\PYZus{}sim\PYZus{}word2vec}\PY{p}{(}\PY{n}{word\PYZus{}to\PYZus{}vec\PYZus{}dict}\PY{p}{,}\PY{l+s+s1}{\PYZsq{}}\PY{l+s+s1}{cat}\PY{l+s+s1}{\PYZsq{}}\PY{p}{,}\PY{l+s+s1}{\PYZsq{}}\PY{l+s+s1}{university}\PY{l+s+s1}{\PYZsq{}}\PY{p}{)}
        \PY{n}{dog\PYZus{}uni\PYZus{}sim}\PY{o}{=}\PY{n}{distsim}\PY{o}{.}\PY{n}{cos\PYZus{}sim\PYZus{}word2vec}\PY{p}{(}\PY{n}{word\PYZus{}to\PYZus{}vec\PYZus{}dict}\PY{p}{,}\PY{l+s+s1}{\PYZsq{}}\PY{l+s+s1}{university}\PY{l+s+s1}{\PYZsq{}}\PY{p}{,}\PY{l+s+s1}{\PYZsq{}}\PY{l+s+s1}{dog}\PY{l+s+s1}{\PYZsq{}}\PY{p}{)}
        
        \PY{k}{print} \PY{l+s+s2}{\PYZdq{}}\PY{l+s+se}{\PYZbs{}n}\PY{l+s+s2}{cosine similarity between cat and dog: }\PY{l+s+s2}{\PYZdq{}}\PY{p}{,} \PY{n}{cat\PYZus{}dog\PYZus{}sim}\PY{p}{,} \PY{l+s+s2}{\PYZdq{}}\PY{l+s+se}{\PYZbs{}n}\PY{l+s+s2}{\PYZdq{}}
        \PY{k}{print} \PY{l+s+s2}{\PYZdq{}}\PY{l+s+s2}{cosine similarity between cat and university: }\PY{l+s+s2}{\PYZdq{}}\PY{p}{,} \PY{n}{cat\PYZus{}uni\PYZus{}sim}\PY{p}{,} \PY{l+s+s2}{\PYZdq{}}\PY{l+s+se}{\PYZbs{}n}\PY{l+s+s2}{\PYZdq{}}
        \PY{k}{print} \PY{l+s+s2}{\PYZdq{}}\PY{l+s+s2}{cosine similarity between university and dog: }\PY{l+s+s2}{\PYZdq{}}\PY{p}{,} \PY{n}{dog\PYZus{}uni\PYZus{}sim}
        \PY{c+c1}{\PYZsh{} write code here to show output (i.e. cosine similarity between these words.)}
        \PY{c+c1}{\PYZsh{} We encourage you to write other functions in distsim.py itself. }
\end{Verbatim}


    \begin{Verbatim}[commandchars=\\\{\}]

cosine similarity between cat and dog:  0.827517295965 

cosine similarity between cat and university:  -0.205394745036 

cosine similarity between university and dog:  -0.190753135501

    \end{Verbatim}

    \textbf{Write your response here:} They do make sense. Cosine similarity
between cat and dog is high and positive, whereas that between cat and
university \& dog and university is negative.

    \subsection{Question 8 (20 points)}\label{question-8-20-points}

Repeat the process you did in the question 6, but now use dense vector
from word2vec. Comment on whether the outputs makes sense. Compare the
outputs of using nearest words on word2vec and the outputs on sparse
context vector (so we suggest you to use the same words in question 6).
Which method works better on the query words you choose. Please brief
explain why one method works better than other in each case.

Not: we used the default parameters of word2vec in
\href{https://radimrehurek.com/gensim/models/word2vec.html}{gensim} to
get word2vec word embeddings.

    \begin{Verbatim}[commandchars=\\\{\}]
{\color{incolor}In [{\color{incolor}1}]:} \PY{k+kn}{import} \PY{n+nn}{distsim}
        \PY{k+kn}{import} \PY{n+nn}{operator}
        \PY{n}{word\PYZus{}list}\PY{o}{=}\PY{p}{[}\PY{l+s+s2}{\PYZdq{}}\PY{l+s+s2}{stock}\PY{l+s+s2}{\PYZdq{}}\PY{p}{,} \PY{l+s+s2}{\PYZdq{}}\PY{l+s+s2}{limited}\PY{l+s+s2}{\PYZdq{}}\PY{p}{,} \PY{l+s+s2}{\PYZdq{}}\PY{l+s+s2}{inning}\PY{l+s+s2}{\PYZdq{}}\PY{p}{,} \PY{l+s+s2}{\PYZdq{}}\PY{l+s+s2}{yellow}\PY{l+s+s2}{\PYZdq{}}\PY{p}{,} \PY{l+s+s2}{\PYZdq{}}\PY{l+s+s2}{saved}\PY{l+s+s2}{\PYZdq{}}\PY{p}{,} \PY{l+s+s2}{\PYZdq{}}\PY{l+s+s2}{paris}\PY{l+s+s2}{\PYZdq{}}\PY{p}{]}
        
        \PY{k}{for} \PY{n}{word} \PY{o+ow}{in} \PY{n}{word\PYZus{}list}\PY{p}{:}
            \PY{n}{dict\PYZus{}cos\PYZus{}sim}\PY{o}{=}\PY{n}{distsim}\PY{o}{.}\PY{n}{stream\PYZus{}cos\PYZus{}sim\PYZus{}calc\PYZus{}word2vec}\PY{p}{(}\PY{n}{word}\PY{p}{)}
            \PY{n}{near\PYZus{}20}\PY{o}{=}\PY{n+nb}{dict}\PY{p}{(}\PY{n+nb}{sorted}\PY{p}{(}\PY{n}{dict\PYZus{}cos\PYZus{}sim}\PY{o}{.}\PY{n}{iteritems}\PY{p}{(}\PY{p}{)}\PY{p}{,} \PY{n}{key}\PY{o}{=}\PY{n}{operator}\PY{o}{.}\PY{n}{itemgetter}\PY{p}{(}\PY{l+m+mi}{1}\PY{p}{)}\PY{p}{,} \PY{n}{reverse}\PY{o}{=}\PY{n+nb+bp}{True}\PY{p}{)}\PY{p}{[}\PY{p}{:}\PY{l+m+mi}{20}\PY{p}{]}\PY{p}{)}\PY{o}{.}\PY{n}{keys}\PY{p}{(}\PY{p}{)}
            
            \PY{k}{print} \PY{l+s+s2}{\PYZdq{}}\PY{l+s+se}{\PYZbs{}n}\PY{l+s+s2}{20 words nearest to }\PY{l+s+s2}{\PYZdq{}}\PY{p}{,} \PY{n}{word}\PY{p}{,} \PY{l+s+s2}{\PYZdq{}}\PY{l+s+s2}{:}\PY{l+s+se}{\PYZbs{}n}\PY{l+s+s2}{ }\PY{l+s+s2}{\PYZdq{}}\PY{p}{,} \PY{n}{near\PYZus{}20}\PY{p}{,} \PY{l+s+s2}{\PYZdq{}}\PY{l+s+se}{\PYZbs{}n}\PY{l+s+s2}{\PYZdq{}}
        \PY{c+c1}{\PYZsh{}\PYZsh{}\PYZsh{}Provide your answer below}
\end{Verbatim}


    \begin{Verbatim}[commandchars=\\\{\}]

20 words nearest to  stock :
  ['shareholders', 'debt', 'investors', 'exchange', 'profit', 'price', 'dollar', 'sale', 'shares', 'sales', 'securities', 'market', 'inflation', 'trading', 'earnings', 'stocks', 'markets', 'investment', 'profits', 'bond'] 


20 words nearest to  limited :
  ['available', 'strong', 'necessary', 'increased', 'providing', 'certain', 'required', 'crucial', 'direct', 'specific', 'critical', 'weak', 'combined', 'significant', 'basic', 'long-term', 'increasing', 'overall', 'positive', 'common'] 


20 words nearest to  inning :
  ['scored', 'mets', 'ball', 'innings', 'n.b.a', 'sixth', 'leg', 'pitcher', 'ninth', 'eighth', 'hitting', 'bout', 'game', 'pistons', 'pitch', 'hole', 'quarter', 'round', 'spurs', 'yankees'] 


20 words nearest to  yellow :
  ['pink', 'blue', 'tree', 'gray', 'leather', 'metal', 'shorts', 'jeans', 'tiny', 'lights', 'bright', 'green', 'shirts', 'smoke', 'wet', 'bucket', 'thick', 'sky', 'wings', 'pants'] 


20 words nearest to  saved :
  ['loved', 'raised', 'save', 'missed', 'lost', 'noticed', 'gotten', 'changed', 'liked', 'paid', 'hurt', 'discovered', 'returned', 'dropped', 'survived', 'collected', 'learned', 'removed', 'brought', 'bought'] 


20 words nearest to  paris :
  ['el', 'del', 'australia', 'italy', 'la', 'madrid', 'hotel', 'royal', 'france', '1960', 'de', 'london', 'japan', '1996', 'germany', 'argentina', 'chelsea', 'chicago', 'spain', 'restaurant'] 


    \end{Verbatim}

    For dense vectors, the outputs make way more sense, hence this method
works the best. In case of \emph{stock}, words closest to it are more
relatable in this case compared to sparse vectors, such as
\emph{shareholders}, \emph{debt}, \emph{investors}, etc. It is the same
case for \emph{inning}, since words closest to it are \emph{scored,
mets, ball}, etc. Unlike dense vectors, sparse vectors contain mostly
zeros. Hence, many words which are actually contextually similar to the
given word are left out when sparse vectors are used. For dense vectors,
very few values are zero, hence contextually similar words do not get
left out.

    \subsection{Question 9 (15 points)}\label{question-9-15-points}

An interesting thing to try with word embeddings is analogical reasoning
tasks. In the following example, it's intended to solve the analogy
question "king is to man as what is to woman?", or in SAT-style
notation,

king : man :: \_\_\_\_ : woman

Some research has proposed to use additive operations on word embeddings
to solve the analogy: take the vector \((v_{king}-v_{man}+v_{woman})\)
and find the most-similar word to it. One way to explain this idea: if
you take "king", get rid of its attributes/contexts it shares with
"man", and add in the attributes/contexts of "woman", hopefully you'll
get to a point in the space that has king-like attributes but the "man"
ones replaced with "woman" ones.

Show the output for 20 closest words you get by trying to solve that
analogy with this method. Did it work?

Please come up with another analogical reasoning task (another triple of
words), and output the answer using the same method. Comment on whether
the output makes sense. If the output makes sense, explain why we can
capture such relation between words using an unsupervised algorithm.
Where does the information come from? On the other hand, if the output
does not make sense, propose an explanation why the algorithm fails on
this case.

Note that the word2vec is trained in an unsupervised manner just with
distributional statistics; it is interesting that it can apparently do
any reasoning at all. For a critical view, see
\href{http://www.aclweb.org/anthology/W/W16/W16-2503.pdf}{Linzen 2016}.

    \begin{Verbatim}[commandchars=\\\{\}]
{\color{incolor}In [{\color{incolor}5}]:} \PY{c+c1}{\PYZsh{} Write code to show output here.}
        \PY{k+kn}{import} \PY{n+nn}{distsim}
        \PY{k+kn}{import} \PY{n+nn}{operator}
        
        \PY{n}{d}\PY{o}{=}\PY{n}{distsim}\PY{o}{.}\PY{n}{king\PYZus{}queen\PYZus{}analogy}\PY{p}{(}\PY{l+s+s1}{\PYZsq{}}\PY{l+s+s1}{king}\PY{l+s+s1}{\PYZsq{}}\PY{p}{,} \PY{l+s+s1}{\PYZsq{}}\PY{l+s+s1}{man}\PY{l+s+s1}{\PYZsq{}}\PY{p}{,} \PY{l+s+s1}{\PYZsq{}}\PY{l+s+s1}{woman}\PY{l+s+s1}{\PYZsq{}}\PY{p}{)}
        
        \PY{n}{near\PYZus{}20}\PY{o}{=}\PY{n+nb}{dict}\PY{p}{(}\PY{n+nb}{sorted}\PY{p}{(}\PY{n}{d}\PY{o}{.}\PY{n}{iteritems}\PY{p}{(}\PY{p}{)}\PY{p}{,} \PY{n}{key}\PY{o}{=}\PY{n}{operator}\PY{o}{.}\PY{n}{itemgetter}\PY{p}{(}\PY{l+m+mi}{1}\PY{p}{)}\PY{p}{,} \PY{n}{reverse}\PY{o}{=}\PY{n+nb+bp}{True}\PY{p}{)}\PY{p}{[}\PY{p}{:}\PY{l+m+mi}{20}\PY{p}{]}\PY{p}{)}\PY{o}{.}\PY{n}{keys}\PY{p}{(}\PY{p}{)}
        
        \PY{k}{print} \PY{l+s+s2}{\PYZdq{}}\PY{l+s+s2}{20 closest words for the analogy }\PY{l+s+se}{\PYZbs{}\PYZdq{}}\PY{l+s+s2}{king : man :: \PYZus{}\PYZus{}\PYZus{}\PYZus{} : woman}\PY{l+s+se}{\PYZbs{}\PYZdq{}}\PY{l+s+s2}{: }\PY{l+s+s2}{\PYZdq{}}\PY{p}{,} \PY{l+s+s2}{\PYZdq{}}\PY{l+s+se}{\PYZbs{}n}\PY{l+s+s2}{\PYZdq{}}\PY{p}{,} \PY{n}{near\PYZus{}20}\PY{p}{,} \PY{l+s+s2}{\PYZdq{}}\PY{l+s+se}{\PYZbs{}n}\PY{l+s+se}{\PYZbs{}n}\PY{l+s+s2}{\PYZdq{}}
        
        
        \PY{n}{d2}\PY{o}{=}\PY{n}{distsim}\PY{o}{.}\PY{n}{king\PYZus{}queen\PYZus{}analogy}\PY{p}{(}\PY{l+s+s1}{\PYZsq{}}\PY{l+s+s1}{france}\PY{l+s+s1}{\PYZsq{}}\PY{p}{,} \PY{l+s+s1}{\PYZsq{}}\PY{l+s+s1}{paris}\PY{l+s+s1}{\PYZsq{}}\PY{p}{,} \PY{l+s+s1}{\PYZsq{}}\PY{l+s+s1}{madrid}\PY{l+s+s1}{\PYZsq{}}\PY{p}{)}
        
        \PY{n}{near\PYZus{}20\PYZus{}d2}\PY{o}{=}\PY{n+nb}{dict}\PY{p}{(}\PY{n+nb}{sorted}\PY{p}{(}\PY{n}{d2}\PY{o}{.}\PY{n}{iteritems}\PY{p}{(}\PY{p}{)}\PY{p}{,} \PY{n}{key}\PY{o}{=}\PY{n}{operator}\PY{o}{.}\PY{n}{itemgetter}\PY{p}{(}\PY{l+m+mi}{1}\PY{p}{)}\PY{p}{,} \PY{n}{reverse}\PY{o}{=}\PY{n+nb+bp}{True}\PY{p}{)}\PY{p}{[}\PY{p}{:}\PY{l+m+mi}{20}\PY{p}{]}\PY{p}{)}\PY{o}{.}\PY{n}{keys}\PY{p}{(}\PY{p}{)}
        
        \PY{k}{print} \PY{l+s+s2}{\PYZdq{}}\PY{l+s+s2}{20 closest words for the analogy }\PY{l+s+se}{\PYZbs{}\PYZdq{}}\PY{l+s+s2}{france : paris :: \PYZus{}\PYZus{} : madrid}\PY{l+s+se}{\PYZbs{}\PYZdq{}}\PY{l+s+s2}{: }\PY{l+s+s2}{\PYZdq{}}\PY{p}{,} \PY{l+s+s2}{\PYZdq{}}\PY{l+s+se}{\PYZbs{}n}\PY{l+s+s2}{\PYZdq{}}\PY{p}{,} \PY{n}{near\PYZus{}20\PYZus{}d2}  
\end{Verbatim}


    \begin{Verbatim}[commandchars=\\\{\}]
20 closest words for the analogy "king : man :: \_\_\_\_ : woman":  
['king', 'singer', 'woman', 'daughter', 'elizabeth', 'clark', 'mother', 'queen', 'royal', 'sister', 'prince', 'mary', 'kim', 'grandma', 'mama', 'lord', 'girl', 'wedding', 'princess', 'husband'] 


20 closest words for the analogy "france : paris :: \_\_ : madrid":  
['ukraine', 'brazil', 'europe', 'korea', 'china', 'afghanistan', 'africa', 'india', 'france', 'arab', 'britain', 'germany', 'canada', 'japan', 'terrorism', 'hamas', 'sunni', 'russia', 'spain', 'iran']

    \end{Verbatim}

    \subsubsection{Textual answer here.}\label{textual-answer-here.}

    Yes, the analogy does work. Some words from list of 20 closest are:
\textbf{elizabeth, queen, royal, mary, princess} All of the words above
allude to a woman of royalty and/or proper name of actual queens, such
as \emph{elizabeth, mary}. Analogy for my analysis is: \textbf{france is
to paris as what is to madrid}? Therefore, in SAT-style notation: france
: paris :: \_\_ : madrid For my analogy, the method does work. The
target answer is \textbf{spain}, a country. The closest words returned
are almost all countries, which is what was desired. Besides
\textbf{spain} being one of the words in the list of 20 closest words,
some of the other countries present are \emph{brazil, ukraine, korea}.
There are also names of continents in the list, like \emph{europe,
africa}. The unsupervised algorithm here determined clusters of words
which are related to each other. In a given tweet, people will talk
about a particular topic only, which leads to usage of contextually
related words in said tweet. For instance, in a tweet about countries to
visit, many words used will be associated with countries or will be
names of countries. If the tweet is about France, there is a good chance
Paris will be mentioned too. So, the unsupervised algorithm obtains
clusters alluding to relationships between words, such as that between
names of countries, a country and its city, etc.


    % Add a bibliography block to the postdoc
    
    
    
    \end{document}
